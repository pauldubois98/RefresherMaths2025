\documentclass[]{article}

\usepackage{amsmath}
\usepackage{amsfonts}
\usepackage{amsthm}
\usepackage{amssymb}
\usepackage{mathrsfs}
\usepackage{stmaryrd}

\newcommand{\Q}{\mathbb{Q}}
\newcommand{\N}{\mathbb{N}}
\newcommand{\Z}{\mathbb{Z}}
\newcommand{\R}{\mathbb{R}}
\newcommand{\Primes}{\mathbb{P}}
\newcommand{\st}{\text{ s.t. }}
\newcommand{\txtand}{\text{ and }}
\newcommand{\txtor}{\text{ or }}
\newcommand{\lxor}{\veebar}

%opening
\title{Notes on Combinatorics}
\author{DSBA Mathematics Refresher 2025}
\date{}

\begin{document}
	
	\maketitle
	
	\begin{abstract}
		
	\end{abstract}
	
	
	Combinatorics is the branch of mathematics dealing with counting, arrangement, and combination of objects.
	It provides the foundation for various concepts in probability, statistics, computer science, and more.
	
	\paragraph{Choosing $n$ Times: Power $n$}
	\noindent\\
	When we choose an item from a set $n$ times \textit{\textbf{with replacement}}, each choice is independent of the previous ones.
	If the set has $k$ distinct elements, the total number of ways to make these choices is given by:
	$ k^n $
	
	\textbf{Example:}
	\noindent\\
	Suppose you have a set of 3 elements $\{a, b, c\}$.
	The number of different sequences of length 4 that can be formed by choosing from this set with replacement is:
	$ 3^4 = 81 $
	
	\paragraph{Ordering $n$ Different Items: Factorial}
	\noindent\\
	When selecting $n$ different items from a set, the number of ways to arrange these $n$ items in order is given by $n!$ (read as "n factorial"), where:
	$$
	n! = n \times (n-1) \times \cdots \times 2 \times 1
	$$
	
	\textbf{Example:}
	\noindent\\
	For a set of 5 distinct items $\{a, b, c, d, e\}$, the number of ways to arrange all 5 items is:
	$$
	5!
	= 5 \times 4 \times 3 \times 2 \times 1
	= 120
	$$
	
	\paragraph{Choosing $k$ from $n$ \textit{(in Order)}}
	\noindent\\
	When choosing $k$ elements from a set of $n$ elements (without replacement) where the order of selection matters (permutations), the number of possible arrangements is given by:
	$$
	P(n, k) = \frac{n!}{(n-k)!}
	$$
	
	\textbf{Example:}
	\noindent\\
	Consider a set of 6 elements $\{a, b, c, d, e, f\}$, and you want to select and arrange 3 elements. The number of different arrangements is:
	$$
	P(6, 3)
	= \frac{6!}{(6-3)!}
	= \frac{6 \times 5 \times 4 \times 3 \times 2 \times 1}{3 \times 2 \times 1}
	= 120
	$$
	
	\paragraph{Choosing $k$ from $n$ \textit{(Without Order)}}
	\noindent\\
	When choosing $k$ elements from a set of $n$ elements where the order of selection does not matter (combinations), the number of possible combinations is given by the binomial coefficient:
	$$
	\binom{n}{k} = \frac{n!}{k!(n-k)!}
	$$
	
	\textbf{Example:}
	\noindent\\
	Given a set of 7 elements $\{a, b, c, d, e, f, g\}$, the number of ways to choose 3 elements without regard to order is:
	$$
	\binom{7}{3}
	= \frac{7!}{3!(7-3)!}
	= \frac{7 \times 6 \times 5 \times 4 \times 3 \times 2 \times 1}{(3 \times 2 \times 1)(4 \times 3 \times 2 \times 1)}
	= 35
	$$
	
	\newpage
	\textbf{{\LARGE Extra: Fundamental Mathematics}}\\
	This part was improvised, as we went through the planned lecture (much) faster than expected.
	
	\section*{Modular Arithmetic}
	\noindent \\
	\textbf{Definition}: Modular arithmetic involves integers where numbers reset to zero after reaching a specified modulus.
	\noindent \\
	\textbf{Real-world Example}: The 12-hour clock system is a common example, where after 12, the next hour is 1.
	\noindent \\
	\textbf{Notation}: $a \equiv b \ (\text{mod} \ m)$
	This means that when $a$ is divided by $m$, it leaves a remainder $b$.

	
	\subparagraph{Basic Concepts}
	\noindent \\
	\textbf{Modulus (m)}: The value at which numbers wrap around.
	\noindent \\
	\textbf{Congruence Relation}: $a \equiv b \ (\text{mod} \ m)$ means $a - b$ is divisible by $m$.
	\noindent \\
	\textbf{Examples}:
	\begin{itemize}
		\item $17 \equiv 5 \ (\text{mod} \ 12)$ because $17 - 5 = 12$, which is divisible by 12.
		\item $20 \equiv 2 \ (\text{mod} \ 9)$ because $20 - 2 = 18$, which is divisible by 9.
	\end{itemize}

	\subparagraph{Exercises}
	Calculate the following:
	\begin{enumerate}
		\item $25 \ (\text{mod} \ 7)$
		\item $37 \ (\text{mod} \ 10)$
		\item $100 \ (\text{mod} \ 13)$
	\end{enumerate}
	
	\subparagraph{Basic Properties}
	\noindent \\
	\textbf{Addition}: If $a \equiv b \ (\text{mod} \ m)$ and $c \equiv d \ (\text{mod} \ m)$, then $(a + c) \equiv (b + d) \ (\text{mod} \ m)$.
	\noindent \\
	\textbf{Subtraction}: $(a - c) \equiv (b - d) \ (\text{mod} \ m)$.
	\noindent \\
	\textbf{Multiplication}: $(a \times c) \equiv (b \times d) \ (\text{mod} \ m)$.
	
	\subparagraph{Link with divisibility}
	\noindent \\
	Modular arithmetic is closely related to the concept of divisibility.
	Specifically, the congruence $a \equiv b \ (\text{mod} \ m)$ implies that the difference $a - b$ is divisible by $m$.
	In other words, there exists an integer $k$ such that $a - b = km$.
	This relationship shows that modular arithmetic can be used to test for divisibility.
	For example, a number $a$ is divisible by $m$ if and only if $a \equiv 0 \ (\text{mod} \ m)$.
	Thus, modular arithmetic provides a powerful tool for solving problems related to divisibility and for simplifying complex calculations by working with remainders.
	
	\subparagraph{Exercises}
	Simplify the following:
	\begin{enumerate}
		\item $(4 + 9) \ (\text{mod} \ 7)$
		\item $(5 \times 8) \ (\text{mod} \ 11)$
		\item Show that $2 \mid k^2 \implies 2 \mid k$ \textit{(hint: separate cases)}		\item Show that $3 \mid k^2 \implies 3 \mid k$ \textit{(hint: separate cases)}
	\end{enumerate}

	\vspace{3cm}
	\section*{Proof by Contradiction}
	\subsection*{$\sqrt{2}$ is not a rational (i.e.: $\sqrt{2} \not\in \mathbb{Q}$)}
	
	We will prove that $\sqrt{2}$ is not a rational number using a proof by contradiction.
	
	\begin{proof}
		Assume, for the sake of contradiction, that $\sqrt{2}$ is rational. Then, we can write $\sqrt{2}$ as a fraction of two integers $\frac{p}{q}$.
		In case $p$ and $q$ have common divisor $k > 1$, we can divide both by that $k$, untill $p$ and $q$ become coprime.
		Hence, we can now assume $\sqrt{2} = \frac{p}{q}$, where $p$ and $q$ are coprime (i.e., they have no common factors other than 1), and $q \neq 0$.
		Hence, we have:
		$$ \sqrt{2} = \frac{p}{q} $$
		Squaring both sides gives:
		$$ 2 = \frac{p^2}{q^2} $$
		Multiplying both sides by $q^2$ results in:
		$$ p^2 = 2q^2 $$
		This implies that $p^2$ is even, and therefore $p$ must also be even (property derived in the modular arithmetic section).
		So, we can write $p = 2k$ for some integer $k$.
		
		Substituting $p = 2k$ into the equation $p^2 = 2q^2$, we get:
		$$ (2k)^2 = 2q^2 $$
		so $$ 4k^2 = 2q^2 $$
		Dividing both sides by 2 yields:
		$$ 2k^2 = q^2 $$
		This implies that $q^2$ is even, and therefore $q$ must also be even.
		
		However, if both $p$ and $q$ are even, then $p$ and $q$ have a common factor of 2, which contradicts our initial assumption that $p$ and $q$ are coprime.
		
		Since our assumption that $\sqrt{2}$ is rational leads to a contradiction, we conclude that $\sqrt{2}$ is not rational.
		Therefore, $\sqrt{2} \not\in \mathbb{Q}$.
	\end{proof}
	
	\subparagraph{Exercises}
	Show, using the same technique that $\sqrt{3} \not\in \mathbb{Q}$.
	
	\vspace{3cm}
	\section*{Proof by Induction}
	Mathematical induction is a powerful technique used to prove statements that are typically formulated in terms of all integers $n \in \N$.
	The principle of induction consists of two main steps:
	\begin{enumerate}
		\item \textbf{Base Case}: Verify that the statement holds for the initial value of $n$, usually $n = 0$ or $n = 1$.
		\item \textbf{Inductive Step}: Assume the statement holds for some arbitrary $n$ (this is called the \emph{inductive hypothesis}), and then prove that the statement holds for $n + 1$.
	\end{enumerate}
	If both steps are successfully completed, the statement is true for all integers $n$ greater than or equal to the base case.
	
	\paragraph{Examples}
	\subparagraph{Sequence $u_{n+1} = \frac{u_n}{2}$ with $u_0 = 1$}
	\noindent\\
	We will prove by induction that $u_n = \frac{1}{2^n}$ for all $n \geq 0$.
	\begin{proof}
		\textbf{Base Case}:
		For $n = 0$,
		$$
		u_0 = 1 = \frac{1}{2^0}.
		$$
		Thus, the statement holds for $n = 0$.
		
		\textbf{Inductive Step}:
		Assume that the statement holds for $n$, i.e., $u_n = \frac{1}{2^n}$.
		We need to prove that the statement holds for $n + 1$.\\
		Using the recurrence relation,
		$$
		u_{n+1} = \frac{u_n}{2}.
		$$
		Substituting the inductive hypothesis $u_n = \frac{1}{2^n}$, we get
		$$
		u_{n+1} = \frac{\frac{1}{2^n}}{2} = \frac{1}{2^{n+1}}.
		$$
		Thus, the statement holds for $n + 1$.\\
		By the principle of mathematical induction, $u_n = \frac{1}{2^n}$ for all $n \geq 0$.
	\end{proof}
	
	\subparagraph*{Sequence $u_1 = 1$, $u_{n+1} = u_n + \frac{1}{2^n}$}
	\noindent\\
	We will prove by induction that $u_n = 2 - \frac{1}{2^{n-1}}$ for all $n \geq 1$.
	\begin{proof}
		\textbf{Base Case}:
		For $n = 1$,
		$$
		u_1 = 1 = 2 - \frac{1}{2^{1-1}} = 2 - 1.
		$$
		Thus, the statement holds for $n = 1$.
		
		\textbf{Inductive Step}:
		Assume that the statement holds for $n$, i.e., $u_n = 2 - \frac{1}{2^{n-1}}$.
		We need to prove that the statement holds for $n + 1$.
		
		Using the recurrence relation,
		$$ u_{n+1} = u_n + \frac{1}{2^n} $$
		Substituting the inductive hypothesis $u_n = 2 - \frac{1}{2^{n-1}}$, we get
		$$ u_{n+1} = \left(2 - \frac{1}{2^{n-1}}\right) + \frac{1}{2^n} = 2 - \frac{1}{2^n} $$
		Thus, the statement holds for $n + 1$.
		
		By the principle of mathematical induction, $u_n = 2 - \frac{1}{2^{n-1}}$ for all $n \geq 1$.
	\end{proof}
	
	\section*{Sum of the First $n$ Integers}
	\noindent\\
	We will prove by induction that the sum of the first $n$ integers is given by
	$$ \sum_{k=0}^n k = \frac{n(n+1)}{2} $$
	\begin{proof}
		\textbf{Base Case}:
		For $n = 0$,
		$$ \sum_{k=0}^0 k = 0 = \frac{0(0+1)}{2} $$
		Thus, the statement holds for $n = 0$.
		
		\textbf{Inductive Step}:
		Assume that the statement holds for $n$, i.e.,
		$$ \sum_{k=0}^n k = \frac{n(n+1)}{2} $$
		We need to prove that the statement holds for $n + 1$.
		
		Consider the sum for $n + 1$:
		$$ \sum_{k=0}^{n+1} k = \sum_{k=0}^n k + (n+1) $$
		Using the inductive hypothesis, we have
		$$ \sum_{k=0}^{n+1} k = \frac{n(n+1)}{2} + (n+1) $$
		Factoring out $n+1$ from the right-hand side,
		$$ \sum_{k=0}^{n+1} k = \frac{n(n+1) + 2(n+1)}{2} = \frac{(n+1)(n+2)}{2} $$
		Thus, the statement holds for $n + 1$.
		
		By the principle of mathematical induction, $\sum_{k=0}^n k = \frac{n(n+1)}{2}$ for all $n \geq 0$.
	\end{proof}
	
	\vspace{5cm}
	\textit{
		These three techniques are the main tools in fundamental mathematics:
		splitting cases (seen in modular arithmetic); contradiction (seen in $\sqrt{2} \not \in \Q$); and induction (examples seen in showing general formula for sequences).
		\\\\
		The rest of the course will be focused on application of mathematics, since you are registered for DSBA, however, it is good to keep in mind that the underlying mathematics exists.
	}
	
\end{document}
